\documentclass{article}

\usepackage{amsmath}

\begin{document}
   Let us make a formula here:
   $f(x) = x^2$ is an example.\\ % Inline formulas are delimited by dollar signs.
   Let's see some formulas, equations and how to align them:

   \begin{equation*}
      1 + 2 = 3 
   \end{equation*}
    
   % It is not possible to enter more than one equation in the equation environment!
   % This also applies to equation* and will fail compilation if more than one equation is present.
   \begin{equation*}
      1 = 3 - 2
   \end{equation*}
    
   % The align environment will align the equations at the ampersand &
   % Single equations have to be seperated by a linebreak \\
   \begin{align*}
      1 + 2 &= 3\\
      1 &= 3 - 2
   \end{align*}

   % Powers, fractions and integrals
   \begin{align*}
      f(x) &= x^2\\
      g(x) &= \frac{1}{x}\\
      F(x) &= \int^a_b \frac{1}{3}x^3\\
      % It is also possible to combine various commands using braces
      \frac{1}{\sqrt{x}}
   \end{align*}
   
   % There are also matrices:
   $\begin{matrix}
      1 & 0\\
      0 & 1
   \end{matrix}$

   % For brackets, this doesn't work, as they would appear too small.
   $[
      \begin{matrix}
         1 & 0\\
         0 & 1
      \end{matrix}
   ]$
   
   % This is the right way of setting the brackets on the sides of a matrix.
   $\left[
      \begin{matrix}
         1 & 0\\
         0 & 1
      \end{matrix}
   \right]$

   % \left and \right work for all sorts of parenthesis:
   $(\frac{1}{\sqrt{x}})$
   $\left(\frac{1}{\sqrt{x}}\right)$

\end{document}