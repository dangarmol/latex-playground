\documentclass{article}

% Used for changing the spacing of your table of contents (and document in general).
\usepackage{setspace}

% Only show a subset of the headings for all sections or for a particular section.
% A value of 0 means that your table of contents will show nothing at all and 5 means, that even subparagraphs will be shown.
% The value has to be set in the preamble of your document and automatically applies to the whole document.
\setcounter{tocdepth}{1} % Show sections
%\setcounter{tocdepth}{2} % + subsections
%\setcounter{tocdepth}{3} % + subsubsections
%\setcounter{tocdepth}{4} % + paragraphs
%\setcounter{tocdepth}{5} % + subparagraphs

\begin{document}

\doublespacing  % Sets the spacing between lines of the following section.
\tableofcontents  % Shows a table of contents with sections, subsections, etc.
\singlespacing  % Back to single spacing.
\newpage

\section{Section}

First section

\subsection{Subsection}

First subsection

% You can also adjust the tocdepth for each section individually.
\addtocontents{toc}{\setcounter{tocdepth}{3}} % Reset to default (3)
\section{Another section}
\subsection{Subsection}
\subsubsection{Subsubsection}

\begin{figure}
   \caption{Dummy figure}
\end{figure}
 
\begin{table}
   \caption{Dummy table}
\end{table}

\begin{appendix}
   \listoffigures  % Shows a list of all figures in the document.
   \listoftables  % Shows a list of all tables in the document.
\end{appendix}

\end{document}